\section*{Sažetak}
\label{sec:summary}
\addcontentsline{toc}{section}{\nameref{sec:summary}}
Cilj ovog završnog rada bio je predstaviti stranu programiranja koja omogućava zabavu korisnika, odnosno igrača kao i samog programera. Prikazuje spajanje različitih znanja unutar jedinstvenog projekta, potiče na razvijanje kreativnosti koja je potrebna kako bi se napravio što zanimljiviji dizajn, animacije, atmosfera, opisali likov i priča, sve to je potrebno kako bi se postigla zainteresiranost, ali i povezanost igrača s igrom. Tijekom izrade igre programer se upoznaje s ljudima iz različitih grana industrije u svrhu prikupljanja potrebnih informacija, bilo to vezano za potrebe izrade igre ili jednostavno povratne informacije osoba koje testiraju igru, istražuju eventualne probleme, daju određene savjete i slično. Konkretno ovim projektom je prikazano kako se s besplatnim resursima može izraditi sustav koji u konačnici pruža sate i sate užitka. U radu je opisan način izrade igrice korištenjem okruženja za izradu igara Unity3D (engl.~\textit{Unity3D game engine}) te programskog jezika C\# koji se koristi u pogonskom alatu Unity3D. Ime igre je Dionyzus, žanr je otvoreni svijet, igra se iz trećeg lica te se radi o 3D igrici. 

Unutar igre je prikazan život Leona te njegov put prema otkrivanju tko stoji iza namještaljke u kojoj njegov najbolji prijatelj umire. Osim odrađivanja zadataka i izazova koji su predstavljeni igraču, moguće je slobodno istraživati grad te obavljati različite poslove neovisno o glavnoj priči.

Ključne riječi: Termin pogonski alat (engl.~\textit{Game engine}) u radu predstavlja kompletan Unity, dakle i editor i C\# biblioteku. Razvojni okvir Unity (engl.~\textit{Unity framework}) predstavlja C\# biblioteku. 3D igra (Igrica u 3 dimenzije), C\# (Programski jezik C sharp), Dionyzus (Naziv igrice).
% Sam rad može poslužiti kao predložak.
