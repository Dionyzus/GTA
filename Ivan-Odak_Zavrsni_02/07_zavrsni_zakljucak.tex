\section{Zaključak}
Ovim projektom predstavljen je rad unutar velikog sustava kao što je Unity3D. Opisano je kako je na jednostavan način moguće izraditi igre korištenjem vrhunskih mogućnosti sučelja za izradu igara u kojem su glavne komponente kao što su fizika, animacije i grafika već implementirane. Opisane su i mogućnosti kako proširiti navedene funkcionalnosti te na koji način su iste upotrebljene u projektu Dionyzus. Ovaj projekt je izrađen tako da je što više komponenti moguće ponovno upotrijebiti u novim projektima sličnih tema, ali isto tako predstavlja osnovne koncepte koje svaka 3D igra treba implementirati. 
Može se reći da su ovim projektom postavljeni temelji igre otvorenog svijeta te iako se radi o kratkoj i jednostavnoj priči, omogućeno je da se na vrlo jednostavan način proširi. To se najviše odnosi na zadatke koje igrač mora odraditi kako bi otkrio priču, ali isto tako i istražio kompletno područje kojem može pristupiti tijekom igranja te na inteligenciju neprijatelja, odnosno likova koji nisu kontrolirani od strane igrača. To može uključivati osluškivanje okoline, inteligentnije pretraživanje područja kao i ponašanje tijekom sukoba.
Unutar skripti cilj je bio postići konzistentnost pisanja k\^oda. To se najviše odnosi na nazive varijabli, metoda, klasa i slično te na održavanje osnovnih koncepata objektno orijentiranog programiranja.
U ovom projektu nisu obrađene kompleksne teme kao što su kompletna optimizacija performansi igre, izrada modela, animacija i slično zato što je projekt izrađen s onim mogućnostima Unityja koje su ponuđene svakom korisniku bez obzira na iskustvo u izradi igara. To se odnosi na preuzimanje gotovih modela iz Unity trgovine s dodacima gdje su svi preuzeti modeli besplatni, što podrazumijeva lošije optimizirane teksture, dimenzije i slično, a izbor objekata je temeljen na potrebama igre.

Za izradu ovog projekta su korištene razne aplikacije kao što su Blender, Audacity i Photoshop.
Animacije su preuzete sa stranice \url{https://www.mixamo.com}, a zvukovi sa \url{https://freesound.org} te su dodatno obrađeni u slučaju potrebe.
Projekt je izrađen uz pomoć \url{https://www.youtube.com} vodiča, gdje su pronađene informacije za izradu kompleksnijih elementa igre. A za snalaženje unutar Unity sučelja je korištena službena dokumentacija.
\begin{thebibliography}{9}
\bibitem{Unity} 
Unity User Manual(2019.*),
\\\texttt{https://docs.unity3d.com/Manual/index.htmll}
\end{thebibliography}